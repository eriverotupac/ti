\section{Semantic Web}
The Semantic Web has many synonyms such as Web 3.0, the Linked Data Web, the Web of Data and intelligent Web transition. It enables data to be linked from a source to any other source and to be understood by computers. In order to have a better  understanding, we should know what is semantic. According to the dictionary, semantic implies meaning or understanding. Thus, the Semantic Web is concerned with the meaning and not the structure of data. Tim Berners Lee (Shadbolt et al., 2006) defined the concept of Semantic web such as The Semantic Web is a Web of actionable information derived from data through a semantic theory for interpreting the symbols. 
As we mentioned before, semantic web was one of the reasons because ontology became the focus of many researchers. That is because the heart of semantic web is the ontology layer as we can appreciate in the Figure ~\ref{fig:sw1} .
This ontology layer consists of the blocks query, ontology and rule. Thus, ontologies in the context of semantic web, describe domain theories for the explicit representation of the semantics of the data. Thus, RDF was extended to RDF schema to include the basic features to needed to define ontologies. But, it was limited and there were different proposals such HTML ontological language (SHOE), the ontology inference layer (OIL) and DAML+OIL. Therefore in 2004, W3C set up a standardization working group to develop a standard for web ontology language resulting in OWL the ontology language standard which is based in Description Logics and KL-One.