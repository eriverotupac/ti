\subsection{UFO}
The Unified Foundational Ontology (UFO) was proposed by Giancarlo Guizzardi and Gerd Wagner in \cite{guizardi:2004}. UFO is based on a GFO and DOLCE. While their main areas of application are the natural sciences and linguistics/cognitive engineering, respectively, the main purpose of UFO is to provide a foundation for conceptual modeling, including agent oriented modeling \cite{guizardi:2005a}.
As summarized in \cite{carbonera:2012}, the most generic UFO concept is Thing, which is specialized in two fundamental entities: Urelement and Set. Urelement is an entity that is not a set. The first distinction that is made between the specializations of Urelement is the fundamental distinction between the categories of Individuals and Universals. Individuals are entities that exist in reality, such as a person, an apple, etc. Universals, in turn, are standard features that can be instantiated in a number of different individuals; it can be understood as high-level abstractions that characterize different classes of individuals. In general, for each of the specializations for Universals, UFO also provides a corresponding specialization for Individuals.
UFO Universals are specialized in Endurant and Perdurant (Event) as shown in Figure ?. This distinction can be understood in terms of the behavior of the individuals of these universals in function of time. Endurant Universals are those whose individuals are always fully present whenever they are present, in the sense that they preserve their identity thru time (e.g., Person, Chair, Planet). Moreover, Perdurant Universals are those whose individuals extend in time accumulating temporal parts, in the sense that they occur in time (e.g., War, Party, Meeting).
Endurant Universals are specialized in Substantial Universals and Moment Universals according to their existential dependency. Substantial Universals are those whose individuals are existentially independent, having spatio-temporal properties and being founded on matter (e.g., Device, Car). Moreover, Moment Universals are those whose individuals are existentially dependent, so that they only can exist in other individuals; they are inherent to these other individuals. Color, for example, whose individuals can only exist in other individuals, is a Moment Universal.

The UFO ontology is divided into three categories: UFO-A which defines the core of UFO, as a comprehensive ontology of endurants; UFO-B which defines terms related to perdurants; and UFO-C which defines terms related to the spheres of intentional and social entities \cite{keylist}(Guizzardi et al., 2007).
In UFO-A, a fundamental distinction in this ontology is between the categories of Particular (Individual) and Universal (Type). The former are real entities possessing a unique identity. The later are pattern of features, which can be realized in a number of different particulars. The upper part of the UFO-A is seen in the Figure ~\ref{fig:ufoa}.

In UFO-B, described in (Guizzardi, Falbo and Guizzardi, 2008), the main focus is Event (Perdurant or Occurrent) which are possible changes from a portion of reality to another, i.e., they may transform reality by changing the state of affairs from one pre-state situation to a post-state situation. Events are existentially dependent on their participants in order to exist. Each participation is itself an event that can be atomic (with no improper parts) or complex (composed of at least two events that can themselves be atomic or complex), but that existentially depends on a single substantial. In this ontology, being atomic and being instantaneous are orthogonal notions, i.e., the former can be time-extended as well as the latter can be composed of multiple (instantaneous) participations (Baiôco et al., 2009). It is appreciated in the Figure ~\ref{fig:ufob}.

Then, UFO-C illustrated in Figure ~\ref{fig:ufoc}, outstands the difference between Agents and Objects. Agents can be physical (e.g., a person) or social (e.g., an organization, a society). Objects can also be further categorized in physical (e.g., a book) or social (e.g., money, language). A Normative Description is a social object which defines one or more rules/norms recognized by at least one social agent. It also defines nominal universals such as social moment universals (e.g., social commitment types), social objects (the crown of the queen of UK) and social roles such as president. Examples of normative descriptions include the Peruvian Constitution, as well as a set of directives on how to perform some actions within an organization.

Social moments are types of intentional moments that are created by the exchange of communicative acts and the consequences of these exchanges (e.g., goal adoption, delegation). A Social Relator is an example of a relator composed of two or more pairs of associated commitments/claims (social moments). An internal or a social commitment is fulfilled by an agent if this agent performs an action that the post-state of this action is a situation that satisfies that commitment (Baiôco et al., 2009). 