\chapter{Ontology}
In this part, we are going to describe the definition of ontology and types of ontologies in section \ref{sec:def}. After that we are going to describe three foundational ontologies in section \ref{sec:fonto} 
\section{Definition}
\label{sec:def}
Ontology begins with the field of philosoph with the classical study of being which remotes to the Greek philosopher Aristotle BC. Aristotle's description ‘the study of being qua being’ involves three things: (1) a study, (2) a subject matter (being), and (3) a manner in which the subject matter is studied (qua being); which introduce the base to the science of metaphysics of first philosophy. So in a philosophical discipline, ontology is characterized by being singular, perspective- and domain independent- and oriented towards making strong claims about the world \citet{orstrom:2005}. However, the term \emph{ontology} (or ontologia) was itself coined in 1613 by two philosophers, Rudolf Göckel (Goclenius), in his Lexicon philosophicum and Jacob Lorhard (Lorhardus), in his Theatrum philosophicumm \citet{smith:2001} .
The importance of ontology in computer science has growth in the last decade gaining a specific role in Artificial Intelligence, Computational Linguistics, and Database theory. Thus, ontology in computer science has begun with the definition of \citet{neches:1991}, who stated that ontology establishes the basic terms and relations comprising the vocabulary of a topic area as well as the rules for combining terms and relations to define extensions to the vocabulary. Later on, Gruber \citeyearpar{gruber:93} defined as an explicit specification of a conceptualization. Based on Gruber’s definition, Borst \citeyearpar{borst:97} defined as a formal specification of a shared conceptualization. After that, Studer's definition merges Borst and Gruber defining ontology as a formal and explicit specification of a shared conceptualization \citet{studer:98}. In the context of information systems, ontologies are seen as engineering artifacts constituted by a specific vocabulary used to describe a certain reality, plus a set of explicit assumptions regarding the intended meaning of the vocabulary words \cite{guarino:1998}. 
There exist many classifications of ontology that \citet{gomez:2004} presents, but in general they can be categorized based on the formalness of the knowledge captured. According to \citet{baade:2003} there are top-level ontology, domain ontology and application ontology which are similar to Guarino's classification \citeyearpar{guarino:1998}, excepted for the task ontology that is not included.

Figure Ontologies Classification (Source Saripalle R 2012)
Top Level Ontologies: describe very general concepts like space, time, etc., which are independent of particular problem or domain.
Domain Ontologies: describe the vocabulary related to a generic domain, by specializing the terms introduced in top level ontology.
Application Ontologies: describe concepts depending on a particular domain, which are often specializations. These concepts often correspond to roles played by domain entities while performing a certain activity.
Some classifications we consider important to mention are the core ontologies and foundational ontologies.
\section{Semantic Web}
The Semantic Web has many synonyms such as Web 3.0, the Linked Data Web, the Web of Data and intelligent Web transition. It enables data to be linked from a source to any other source and to be understood by computers. In order to have a better  understanding, we should know what is semantic. According to the dictionary, semantic implies meaning or understanding. Thus, the Semantic Web is concerned with the meaning and not the structure of data. Tim Berners Lee (Shadbolt et al., 2006) defined the concept of Semantic web such as The Semantic Web is a Web of actionable information derived from data through a semantic theory for interpreting the symbols. 
As we mentioned before, semantic web was one of the reasons because ontology became the focus of many researchers. That is because the heart of semantic web is the ontology layer as we can appreciate in the Figure ~\ref{fig:sw1} .
This ontology layer consists of the blocks query, ontology and rule. Thus, ontologies in the context of semantic web, describe domain theories for the explicit representation of the semantics of the data. Thus, RDF was extended to RDF schema to include the basic features to needed to define ontologies. But, it was limited and there were different proposals such HTML ontological language (SHOE), the ontology inference layer (OIL) and DAML+OIL. Therefore in 2004, W3C set up a standardization working group to develop a standard for web ontology language resulting in OWL the ontology language standard which is based in Description Logics and KL-One.
\section{Foundational Ontologies\label{sec:fonto}}
Foundational ontologies are meta ontologies used to evaluate the concept modelling. There is a misunderstanding when we talk about foundational ontologies and top level ontologies.  Some authors said that they are the same. However, as we are going to see, the foundational ontology is used to model any kind of ontology. Let’ see some of them.
\subsection{dolce}

DOLCE (Descriptive Ontology for Linguistic and Cognitive Engineering) makes a variety of important distinctions that are useful for upper ontologies. It was developed in a principal way using OntoClean methodology to help ensure correct consistent distinctions. OntoClean (Guarino and Welty, 2009) presents four basic properties that an entity should have which are essence, identity, unity and dependence. Within essence property we have rigidity which states that an entity is essential to all its possible instances. The identity property refers to recognize the entity based on the essential properties. However, it could be hard to determine. The unity property refers to specify if properties have wholes as instances. Dependence occurs when individuals of a property depend on the existence of individuals of other property in order to exist.
DOLCE is an ontology of particulars, is based on a fundamental distinction between endurants and perdurants entities. The difference between these is related to the behavior in time. Endurants are wholly present at any time they are present. On the other hand, perdurants extended in time by accumulating different temporal parts. The relation between endurants and perdurants is that of participation: an endurant lives in time by participating in a perdurant. The structute of the foundational ontology is presented in Figure ~\ref{fig:dolce}

\subsection{UFO}
The Unified Foundational Ontology (UFO) was proposed by Giancarlo Guizzardi and Gerd Wagner in \cite{guizardi:2004}. UFO is based on a GFO and DOLCE. While their main areas of application are the natural sciences and linguistics/cognitive engineering, respectively, the main purpose of UFO is to provide a foundation for conceptual modeling, including agent oriented modeling \cite{guizardi:2005a}.
As summarized in \cite{carbonera:2012}, the most generic UFO concept is Thing, which is specialized in two fundamental entities: Urelement and Set. Urelement is an entity that is not a set. The first distinction that is made between the specializations of Urelement is the fundamental distinction between the categories of Individuals and Universals. Individuals are entities that exist in reality, such as a person, an apple, etc. Universals, in turn, are standard features that can be instantiated in a number of different individuals; it can be understood as high-level abstractions that characterize different classes of individuals. In general, for each of the specializations for Universals, UFO also provides a corresponding specialization for Individuals.
UFO Universals are specialized in Endurant and Perdurant (Event) as shown in Figure ?. This distinction can be understood in terms of the behavior of the individuals of these universals in function of time. Endurant Universals are those whose individuals are always fully present whenever they are present, in the sense that they preserve their identity thru time (e.g., Person, Chair, Planet). Moreover, Perdurant Universals are those whose individuals extend in time accumulating temporal parts, in the sense that they occur in time (e.g., War, Party, Meeting).
Endurant Universals are specialized in Substantial Universals and Moment Universals according to their existential dependency. Substantial Universals are those whose individuals are existentially independent, having spatio-temporal properties and being founded on matter (e.g., Device, Car). Moreover, Moment Universals are those whose individuals are existentially dependent, so that they only can exist in other individuals; they are inherent to these other individuals. Color, for example, whose individuals can only exist in other individuals, is a Moment Universal.

The UFO ontology is divided into three categories: UFO-A which defines the core of UFO, as a comprehensive ontology of endurants; UFO-B which defines terms related to perdurants; and UFO-C which defines terms related to the spheres of intentional and social entities \cite{keylist}(Guizzardi et al., 2007).
In UFO-A, a fundamental distinction in this ontology is between the categories of Particular (Individual) and Universal (Type). The former are real entities possessing a unique identity. The later are pattern of features, which can be realized in a number of different particulars. The upper part of the UFO-A is seen in the Figure ~\ref{fig:ufoa}.

In UFO-B, described in (Guizzardi, Falbo and Guizzardi, 2008), the main focus is Event (Perdurant or Occurrent) which are possible changes from a portion of reality to another, i.e., they may transform reality by changing the state of affairs from one pre-state situation to a post-state situation. Events are existentially dependent on their participants in order to exist. Each participation is itself an event that can be atomic (with no improper parts) or complex (composed of at least two events that can themselves be atomic or complex), but that existentially depends on a single substantial. In this ontology, being atomic and being instantaneous are orthogonal notions, i.e., the former can be time-extended as well as the latter can be composed of multiple (instantaneous) participations (Baiôco et al., 2009). It is appreciated in the Figure ~\ref{fig:ufob}.

Then, UFO-C illustrated in Figure ~\ref{fig:ufoc}, outstands the difference between Agents and Objects. Agents can be physical (e.g., a person) or social (e.g., an organization, a society). Objects can also be further categorized in physical (e.g., a book) or social (e.g., money, language). A Normative Description is a social object which defines one or more rules/norms recognized by at least one social agent. It also defines nominal universals such as social moment universals (e.g., social commitment types), social objects (the crown of the queen of UK) and social roles such as president. Examples of normative descriptions include the Peruvian Constitution, as well as a set of directives on how to perform some actions within an organization.

Social moments are types of intentional moments that are created by the exchange of communicative acts and the consequences of these exchanges (e.g., goal adoption, delegation). A Social Relator is an example of a relator composed of two or more pairs of associated commitments/claims (social moments). An internal or a social commitment is fulfilled by an agent if this agent performs an action that the post-state of this action is a situation that satisfies that commitment (Baiôco et al., 2009). 
\subsection{GFO}
General Formal Ontology (GFO) uses the General Ontology Language (GOL) to defines the three-layered meta-ontological architecture comprised of a basic level consisting of all relevant GFO categories, a meta-level, called abstract core level, containing meta-categories over the basic level, and an abstract top level including set and item (urelement) as the only meta-meta-categories (Herre et al., 2006).
In the GFO, it is outstand the difference between an urelement and a set. It says that Urelements are divided into individuals and universals. Individuals belong to the realm of concrete entities, which means that they exist within the confines of space and time. Universals, in contrast, are entities that can be instantiated simultaneously by a multiplicity of different individuals that are similar in given respects. As concerns individuals, they can be classified into moments, substances, chronoids, topoids and situoids (capture chronoids and topoids).


%http://www.omg.org/news/meetings/tc/dc-13/special-events/semantic-pdfs/T1-1-Kendall.pdf