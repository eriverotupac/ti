\subsection{GFO}
General Formal Ontology (GFO) uses the General Ontology Language (GOL) to defines the three-layered meta-ontological architecture comprised of a basic level consisting of all relevant GFO categories, a meta-level, called abstract core level, containing meta-categories over the basic level, and an abstract top level including set and item (urelement) as the only meta-meta-categories (Herre et al., 2006).
In the GFO, it is outstand the difference between an urelement and a set. It says that Urelements are divided into individuals and universals. Individuals belong to the realm of concrete entities, which means that they exist within the confines of space and time. Universals, in contrast, are entities that can be instantiated simultaneously by a multiplicity of different individuals that are similar in given respects. As concerns individuals, they can be classified into moments, substances, chronoids, topoids and situoids (capture chronoids and topoids).