\subsection{dolce}

DOLCE (Descriptive Ontology for Linguistic and Cognitive Engineering) makes a variety of important distinctions that are useful for upper ontologies. It was developed in a principal way using OntoClean methodology to help ensure correct consistent distinctions. OntoClean (Guarino and Welty, 2009) presents four basic properties that an entity should have which are essence, identity, unity and dependence. Within essence property we have rigidity which states that an entity is essential to all its possible instances. The identity property refers to recognize the entity based on the essential properties. However, it could be hard to determine. The unity property refers to specify if properties have wholes as instances. Dependence occurs when individuals of a property depend on the existence of individuals of other property in order to exist.
DOLCE is an ontology of particulars, is based on a fundamental distinction between endurants and perdurants entities. The difference between these is related to the behavior in time. Endurants are wholly present at any time they are present. On the other hand, perdurants extended in time by accumulating different temporal parts. The relation between endurants and perdurants is that of participation: an endurant lives in time by participating in a perdurant. The structute of the foundational ontology is presented in Figure ~\ref{fig:dolce}
