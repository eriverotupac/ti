\chapter{geology}
Before start describing the concept of geology, we introduce the concept of Geoscience which 
Geoscience includes all the sciences (geology, geophysics, geochemistry) that study the structure, evolution and dynamics of the planet Earth and its natural mineral and energy resources. Geoscience investigates the processes that have shaped the Earth through its 4600 million year history and uses the rock record to unravel that history - it is concerned with the real world beyond the laboratory and has direct relevance to the needs of society. Geoscience is also called earth science. 

\section{Petroleum Geology}
Before define petroleum geology, we will describe the concept of Petroleum. Petroleum comes from the Latin petra, rock or stone and oleum, oil. It occurs widespread in the earth as gas liquid, semi-solid, or solid, or in more that one of this states at a single place. Chemically Petroleum is an extremely complex mixture of hydrocarbon compounds, with minor amounts of nitrogen, oxygen and sulphur as impurities (Lavorsen, 1958). Petroleum in their different states  constitutes an important strategic material which is connected with the vitals and safety of the national economy, and the supplier selections are related to the safety of petroleum production and supply.
The term "petroleum geology" is discussed by some authors that thought that the correct term should be geology of petroleum. However, the person who applied the geological principles to finding petroleum is called a petroleum geologist. Thus, petroleum geology or geology of petroleum is the application of  geology (the study of the physical Earth and the process the form it) to the exploration for and production of oil and gas (Selley 1998).  It is important to mention, that geology is firmly based on physics, chemistry and biology.  Within geology the different sub fields are used to the exploration and production of petroleum as seen in the Figure ~\ref{fig:petrogeo}


Petrology is a subfield of geology that involves the study of rock, their composition, and the process that formed. Within this field, we have a subfield called experimental petrology that involves  a synthesis, fusion, and/or crystallization of minerals and rocks in  laboratory to understand the physical and chemical conditions in which the minerals were formed and are stables. The petrology is strictly joined to geochemistry, through the two chemical elements contained in rocks, and with the geochronology, which techniques provided the rocks age.  Geochemistry deal with the quantity, distribution, and migration of chemical elements (and their isotopes) in the earth and planetary materials. These chemical elements are contained in minerals, rocks and ground. Furthermore, flows within the earth transport chemical components from one reservoir to other. Understand the interchange between those different reservoirs requires knowledge of  mineral and how they react between them. One sub field of Geochemistry is geochemistry of isotope which involves the use of isotopes found in minerals to determine the geological age of them. 
Paleontology \cite{stearn:1989} is an historial science. From fossils paleontologists reconstruct long-dead organisms and the world in which they lived. They can’t experiment with the materials they study, by they examine the producs of past events and try to reconstruct what happened.  Paleontology don't deal with repeateable events.
Geophysic is the study of physic of the earth