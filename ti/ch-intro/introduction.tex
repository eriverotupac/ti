\chapter{INTRODUCTION}
Nowadays many fields such as business intelligence, software engineering, medicine, petroleum engineering and biology are implementing ontologies to ensure better sustainability of the knowledge implemented in systems.  We consider that ontology is a key aspect in the interchanging of systems information and also it lets humans and computers share and reuse information. Gruber (1993) define ontology as explicit specification of a conceptualization. N. Guarino(1998) refines Gruber definition as "a logical theory accounting for the intending meaning of a formal vocabulary. The intended models of a logical language using such a vocabulary are constrained by its ontological commitment. An ontology indirectly reflects this commitment (and the underlying conceptualization) by approximating these intended models". Also, it is important to mention that ontology is language dependent, while a conceptualization is language independent. One of the reasons that ontologies are the focus of many researchers is due to the current growth of semantic web. Thus, today we have the semantic web that represents the next major evolution in connecting information. This engenders a completely different outlook on how storing, querying and displaying information might be approached. Ontology is being used to develop better knowledge models, which are well founded and easy to integrate. However, we need to have a solid background and a good methodology for developing ontologies. Thus, N. Guarino and C. Welty presented the OntoClean which is a methodology for validating the ontological adequacy and logical consistensy of taxonomic relations. Also, Guizzardi (2005) has proposed Ontological Foundations for Structural Conceptual Models, and a few years later, Guizzardi and Wagner (2008) presented a foundational ontology to provide real world semantics and sound modeling guidelines using conceptual modeling called Unified Foundational Ontology(UFO). UFO is based on GFO/GOL  and DOLCE/OntoClean.UFO has three parts UFO-A(endurants), UFO-B(perdurants) and UFO-C(ontology of Social and Intentional Entitties).

Petroleum geology is the application of  geology (the study of the physical Earth and the process the form it) to the exploration for and production of oil and gas (Selley 1998). Geology itself is firmly based on chemistry, physics, and biology. From the combination of these pure sciences we have geochemistry, structural geology, sedimentology, petrography, stratigraphy, paleontology, geophysics exploration and logging, which are part of geology. It is important to mention that petroleum geology is only one aspect of petroleum exploration and production. However, it plays an important role. As an example,  Abel (2001) in her PhD Thesis developed an ontology for sedimentary petrography used in the system PETROLEGE, which helps in the description and interpretation of sedimentary rocks. This will be studied in our review. 

In semantic web applied to Geoscience, we have the Earth and Environmental Terminology (SWEET) which is a top level ontology and it is been used recently in many projects as the basis for extend it to mineralogy, hydrology and other areas of earth science. Ajay and Hassan (2008) describe the SWEET ontology and outstand the importance of maintain consistency with other geology ontologies. Thus, we should develop geology ontologies considering, in the future, the integration with other models. 

Many organizations invest huge amount of money in finding good quality reservoirs and within petroleum exploration it used different techniques in the discovery of good quality reservoirs. Thus, our motivation for doing this review is to have in mind the different kinds of models and ontologies that exist for petroleum geology in order to find similarities and identify the way of integrating these models to support a good petroleum exploration prediction.

The main purpose of our review consists of an overview of the state of the art in ontology and conceptual models applied to petroleum geology. Also, it will contain an analysis of the ontologies founded previously from the point of view of OntoClean and UFO. Concluding with a comparison of the ontologies and showing if exists similar geological concepts between them.

During the description of each taxonomy,standard or ontology, it will be explain the reason for the development, . The criteria of analysis that will be described in the comparison will be the purpose,